\documentclass[conference]{IEEEtran}

\hyphenation{op-tical net-works semi-conduc-tor}

\begin{document}

\title{Un-censorable, DDoS-proof, Fault-tolerant, Decentralised \& Open Domain Name System}

\author{\IEEEauthorblockN{Harsh Vakharia}
\IEEEauthorblockA{harshjv@gmail.com}}

\maketitle

\begin{abstract}
Domain Name System is currently monopolised by the Internet Corporation of Assigned Names and Numbers (ICANN). In recent times, DNS system has been attacked several times and left internet vulnerable. A lot of websites `insert sources` suffered by such attacks and it got everyone thinking of finding a better solution for domain name resolution.
\end{abstract}

\vspace{\baselineskip}

\begin{IEEEkeywords}
Domain Name System, Blockchain, Ethereum, IPFS, Solidity, Smart Contracts, DNS proxy server, Domain Name Registrar, Web3
\end{IEEEkeywords}


\IEEEpeerreviewmaketitle

\section{Introduction}


\section{Smart Contracts}
Solidity is a higher level programming language for Ethereum Virtual Machine that allows deployment and execution of smart contracts without requiring any centralised or trusted parties. 

\section{IPFS}
\section{Compatibility with existing DNS}
\section{Ethereum}
\section{Record Structure}
DNS 

unlimited free reads
paid writes

\section{Example}
hubble, nebulis

\section{Architecture}
The architecture of Open DNS is built atop Ethereum Blockchain. Ethereum Blockchain provides \#insert features of eth-bc and is replicated in 1000s of computers worldwide. Smart contracts, they run in ethereum virtual machine and using Web3, a normal computer program can interact with ethereum eco-system. For brevity, demonstrated smart contract supports fewer record types such as A, NS, CNAME, SOA, PTR, MX, TXT, AAAA, SRV, NAPTR, OPT, SPF and TLSA.

\section{Errors}
\noindent Internal compiler error: Stack too deep, try removing local variables. \\
VM Exception: out of gas

\section{Conclusion}
\section{Making DDoS proof/caching/tendermint}
\section{Problem with domain expiration}

\begin{thebibliography}{1}

\bibitem{IEEEhowto:kopka}
H.~Kopka and P.~W. Daly, \emph{A Guide to \LaTeX}, 3rd~ed.\hskip 1em plus
  0.5em minus 0.4em\relax Harlow, England: Addison-Wesley, 1999.

\end{thebibliography}

\end{document}
